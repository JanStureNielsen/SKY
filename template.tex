% math package (for align environment)
\usepackage{amsmath}
\usepackage{amsfonts}

% new environments to define objects
\newtheorem{definition}{Definition}
\newtheorem{remark}{Remark}
\newtheorem{exercise}{Exercise}
\newenvironment{numexample}{\noindent\textsc{Numerical Example}:\it}{\rm}
\newtheorem{theorem}{Theorem}
\newtheorem{lemma}{Lemma}
\newtheorem{corollary}{Corollary}
\newenvironment{proof}{\textsc{Proof}:}{\rule{1ex}{1ex}}
\newtheorem{assumption}{Assumption}

% remarks
\usepackage[colorinlistoftodos]{todonotes}
\newcommand{\fixmeJ}[2][]{\todo[inline, color=green!20, #1]{#2}}
\newcommand{\fixmeS}[2][]{\todo[inline, color=yellow!40, #1]{#2}}

% package to generate commands with double letters in maths
\usepackage{dsfont}
\usepackage{bbm}
\newcommand{\esp}[1]{\mathbb{E}[ #1 ]}
\newcommand{\var}[1]{\mathbb{V}[ #1 ]}
\newcommand{\cov}[1]{\mathbb{C}[ #1 ]}
\newcommand{\un}[1]{\mathds{1}[ #1 \geq0]}
\newcommand{\unn}[1]{\mathds{1}[ #1 <0]}
\newcommand{\uns}[1]{\mathbb{1}[ #1 ]}
\newcommand\Ind{\protect\mathpalette{\protect\independenT}{\perp}}
\def\independenT#1#2{\mathrel{\setbox0\hbox{$#1#2$}\copy0\kern-\wd0\mkern4mu\box0}} 
\newcommand{\plim}[1]{\text{plim}_{ #1 \rightarrow \infty}}
\newcommand{\plims}{\text{plim}}
\newcommand{\partder}[2]{\frac{\partial #1}{\partial #2}}
\DeclareMathOperator{\diag}{diag}
